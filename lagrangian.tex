% Standalone document
\documentclass[notes.tex]{subfiles}
\begin{document}
%%%%%%%%%%%%%%%%%%%%%%%%%%%%%%%%%%%%%%%%%%%%%%%%%%%%%%%
%%%%%%%%%%%%%%%%%%%%%%%%%%%%%%%%%%%%%%%%%%%%%%%%%%%%%%%
\chapter[Construction of a low-energy SUSY Lagrangian]{Construction of a low-energy supersymmetric Lagrangian}
\label{chap:lagrangian}
%%%%%%%%%%%%%%%%%%%%%%%%%%%%%%%%%%%%%%%%%%%%%%%%%%%%%%%
%%%%%%%%%%%%%%%%%%%%%%%%%%%%%%%%%%%%%%%%%%%%%%%%%%%%%%%
We would now like to construct models, in the form of a field theory Lagrangians, that are invariant under supersymmetry transformations, much in the same way that the Standard Model Lagrangian is invariant under Poincar\'e transformations. However, just as for ordinary quantum field theory Lagrangians we will need to be able to impose gauge invariance for a choice of gauge group, and we want to limit the models to models that are {\it renormalisable}, {\it i.e.}\ models where any infinities can be cancelled by a finite number of counter-terms. Along the way we will also find a way to deal with the problem of equal masses for the particles in an irreducible representation -- in other words, where are all the supersymmetric partners of the Standard Model particles? -- taking inspiration from the Standard Model Higgs mechanism using spontaneous symmetry breaking.



%%%%%%%%%%%%%%%%%%%%%%%%%%%%%%%%%%%%%%%%%%%%%%%%%%
\section[Supersymmetry invariant Lagrangians]{Supersymmetry invariant Lagrangians and actions}
%%%%%%%%%%%%%%%%%%%%%%%%%%%%%%%%%%%%%%%%%%%%%%%%%%
As should be well known the relativistic field theory {\bf action}
\begin{equation}
S\equiv \int_R d^4x\,\mathcal{L},
\end{equation}
is invariant under supersymmetry transformations if this transforms the Lagrangian by a total derivative term $\mathcal{L} \to \mathcal{L}' = \mathcal{L} + \partial^\mu f(x)$, where $f(x) \to 0$ on the surface $S(R)$ of the integration region $R$. The question then becomes: how can we construct a Lagrangian from superfields with this property? 

An infinitesimal {\bf supersymmetry transformation} of a function on superspace $F(x,\theta,\bar\theta)$ can be written as
\begin{equation}
F'(x,\theta,\bar\theta)=\exp(i\alpha Q + i\bar{\alpha}\bar{Q}) F(x,\theta,\bar\theta)\simeq F(x,\theta,\bar\theta) +  (i\alpha Q + i\bar{\alpha}\bar{Q}) F(x,\theta,\bar\theta),
\end{equation}
where $\alpha^A$ and $\bar{\alpha}_{\dot{A}}$ are the infinitesimal parameters of the transformation. 
If these parameters are constant then we say that this is a \textbf{global} supersymmetry transformation which is what we are (mostly) going to concern ourselves with in these notes. Replacing $\alpha \to \alpha(x)$ gives a \textbf{local} supersymmetry transformation.\footnote{Allowing for local supersymmetry transformations leads to a gauge theory of supersymmetry called  supergravity, which can incorporate gravity in a natural way.} 
Here we have ignored the action of the $P_\mu$ operators since they commute with all the $Q$s, so that this part of the group acts independently, and only results in a translation of the Minkowski coordinate. The total change in the function under the supersymmetry transformation is $\delta_sF=F'-F$, where
\begin{equation}
\delta_s = i(\alpha Q+\bar\alpha\bar Q).
\label{eq:SUSY_trans}
\end{equation}

We can show that the highest order component fields in $\theta$ and $\bar{\theta}$ of a superfield always transform as a total derivative under $\delta_s$, {\it e.g.}\ for the general superfield the highest order component field $d(x)$ transforms as\footnote{A word of warning: we really have to be slightly more careful here with the supersymmetry transformation since the superfields are operator valued functions acting on a state-space. A unitary transformation $U|\psi\rangle$ of the state implies a unitary transformation $U^\dagger AU$ of an operator $A$ on that state.}
\begin{equation}
\delta_s d(x) = d'(x)-d(x)= \frac{i}{2}(\partial_\mu \psi(x)\sigma^\mu\bar{\alpha} - \partial_\mu \bar{\lambda}(x) \sigma^\mu\alpha),
\end{equation}
and the same is naturally true for the $D$ component-field of a vector superfield. For a scalar superfield it is the $F$-field which has this property:
\begin{equation}
\delta_s F(x) = -i\sqrt{2}\partial_\mu\psi(x)\sigma^\mu\bar{\alpha}.
\end{equation}

These highest power $\theta$-components can be isolated by using the projection property of integration in Grassmann calculus writing
\[S=\int_R d^4x \int d^4\theta\, \mathcal{L},\]
where the Lagrangian density $\mathcal{L}$, which is a function of superfields, is guaranteed to give a supersymmetry invariant action. Note that this constitutes a redefinition of what we mean by the Lagrangian $\mathcal{L}$ when working with superfields. We should in particular be careful when counting the dimension of terms. We have, see Section \ref{sec:superspace}, $[\theta] = M^{-1/2}$, which, since $\int d\theta \theta=1$, leads to $[\int d\theta]=M^{1/2}$. We then have $[\int d^4\theta] = M^2$. Since we must have $[\int d^4\theta\, \mathcal{L}] = M^4$ for the action to be dimensionless, we need $[\mathcal{L}]=M^2$, which is different from the standard field theory Lagrangian with dimension $M^4$.

We can now write down the generic form for a supersymmetric Lagrangian consisting of scalar superfields $\Phi_i$, where the indices indicate the highest power of $\theta$ in the term:
\[\mathcal{L} = \mathcal{L}_{\theta \theta \bar{\theta} \bar{\theta}}(\Phi_i,\Phi_i^\dagger) + \bar{\theta}\bar{\theta}\mathcal{L}_{\theta\theta}(\Phi_i)+ \theta\theta \mathcal{L}_{\bar{\theta}\bar{\theta}}(\Phi_i^\dagger) .\]
Here $\mathcal{L}_{\theta\theta}$ ($\mathcal{L}_{\bar{\theta}\bar{\theta}}$)  is a function of {\it only}  left-handed (right-handed) scalar superfields where we project out the $F$-field by multiplying by $\bar{\theta}\bar{\theta}$ ($\theta\theta$) and integrating over all the $\theta$. The function form is limited to {\bf holonomic functions} which means that the term is itself a scalar superfield. These two terms form what we call the {\bf superpotential}.
Meanwhile, the $\mathcal{L}_{\theta \theta \bar{\theta} \bar{\theta}}$ term is a real valued function of the scalar superfields where we project out the $d$-field, called the {\bf Kähler potential}. Possible terms include $\Phi_i^\dagger\Phi_i$, but not $\Phi_i+\Phi_i^\dagger$, since this would belong in the superpotential.

The requirement of renormalisability of the resulting quantum field theory puts further restrictions on the fields in $\mathcal{L}$. We can at most have three factors of scalar superfields in each term of the superpotential, and two factors of scalar superfields in the Kähler potentia,  for details see {\it e.g.}\ Wess \& Bagger \cite{Bagger:1983mv}.\footnote{The argument is similar to ordinary quantum field theory, terms with couplings with negative mass dimension is forbidden.} 
Since the total action must be real, the (almost) most general supersymmetry Lagrangian that can be written in terms of scalar superfields is:
\[\mathcal{L} = \Phi^\dagger_i\Phi_i + \bar{\theta}\bar{\theta}W[\Phi_i] + \theta\theta W[\Phi^\dagger_i].\]
Here the first term is called the {\bf kinetic term},\footnote{The constant in front of this term can always be chosen to be one because we can rescale the whole Lagrangian. It is possible to start from more general Kähler potentials, but this is mostly beyond the scope of these notes, although we will return to modify this term to account for gauge invariance.} and $W$ is the symbol for the {\bf superpotential}, which by renormalisability is restricted to
\begin{equation}
W[\Phi] = g_i \Phi_i + m_{ij}\Phi_i \Phi_j + \lambda_{ijk} \Phi_i \Phi_j \Phi_k.
\end{equation}
This means that to specify a supersymmetric Lagrangian we only need to specify the superfield content of the model and the form of the superpotential. 

Dimension counting, starting from $[\mathcal{L}]=M^2$ and $[\Phi]=1$, means that the superpotential terms must each be $M^3$, for the couplings this gives $[g_i]=M^2$, $[m_{ij}]=M$ and $[\lambda_{ijk}]=1$. Notice also that $m_{ij}$ and $\lambda_{ijk}$ must be symmetric since the superfields commute.



%%%%%%%%%%%%%%%%%%%%%%%%%%%%%
\section[Albanian gauge theories]{Abelian gauge theories}
%%%%%%%%%%%%%%%%%%%%%%%%%%%%%
We will of course require not only a supersymmetry invariant Lagrangian, but also a gauge invariant Lagrangian, and we start with the requirements for an abelian gauge group. 

Let us first look at the transformation of the superpotential $W$ under the gauge transformation of the scalar superfields in (\ref{eq:scalar_supergauge}):
\[W[\Phi] \to W[\Phi'] = t_ie^{iq_i\Lambda }\Phi_i + m_{ij}e^{i(q_i+q_j)\Lambda}\Phi_i\Phi_j + \lambda_{ijk}e^{i(q_i+q_j+q_k)\Lambda}\Phi_i\Phi_j\Phi_k.\]
Requiring gauge invariance, $W[\Phi] = W[\Phi']$, we must have:
\begin{eqnarray}
q_i &=& 0 ~~\text{or}~t_i = 0, \\
q_i+q_j &=& 0 ~~\text{or}~~m_{ij} = 0,  \\
q_i+q_j+q_k &=& 0  ~~\text{or}~~ \lambda_{ijk}=0.
\end{eqnarray}
This puts great restrictions on the form of the superpotential and the charge assignments of the superfields, as in ordinary gauge theories. 

What then about the kinetic term? This transforms as
\[\Phi^\dagger_i\Phi_i \to \Phi^\dagger_i e^{-iq_i\Lambda^\dagger }e^{iq_i\Lambda }\Phi_i = e^{iq_i(\Lambda - \Lambda^\dagger)}\Phi^\dagger_i\Phi_i.\]
As in  ordinary gauge theories we can introduce a {\bf gauge compensating vector (super)field} $V$ with the appropriate gauge transformation to make the kinetic term invariant under supersymmetry transformations. We write the kinetic term instead as $\Phi^\dagger_i e^{q_iV}\Phi_i$, which gives us an invariant term:\footnote{In case you were worried: we can use the WZ gauge to show that the new kinetic term $ \Phi^\dagger_ie^{q_iV}\Phi_i$ has no term with dimension higher then four, and is thus renormalisable.}
\[\Phi^\dagger_ie^{q_iV}\Phi_i \to \Phi^\dagger_i e^{-iq_i\Lambda^\dagger }e^{q_i(V - i\Lambda + i\Lambda^\dagger)}e^{iq_i\Lambda}\Phi_i = \Phi^\dagger_ie^{q_iV}\Phi_i.\]

This definition of gauge transformation can be shown to recover the Standard  Model {\bf minimal coupling} for the component fields through the covariant derivative
\[D_\mu^i = \partial_\mu -\frac{i}{2}q_iV_\mu,\]
where $V_\mu$ is the vector component field of the vector superfield. The factor of a half here is admittedly a little odd, but is in fact related to how we (in these notes) defined the supergauge transformations. We can now simply redefine the charge, or go back and fiddle with the numerical factor in front of the vector superfields in the kinetic terms. Your choice!
% TODO: Change conventions to $\Phi e^{2V}$ to avoid this problem?


%%%%%%%%%%%%%%%%%%%
\section{Non-Abelian gauge theories}
%%%%%%%%%%%%%%%%%%%
How do we extend the above to deal with the much more complicated non-abelian gauge theories? Let us take a gauge group $G$ with the Lie algebra of group generators $t_a$ 
\begin{equation}
[t_a, t_b] = if_{ab}{}^c t_c,\label{eq:Liealgebrawta}
\end{equation} 
where $f_{ab}{}^{c}$ are the structure constants. We recap that for an element $g$ in the group $G$ we want to write down a unitary representation $U(g)$ that transforms a vector $\Psi$  in the representation by $\Psi \to \Psi' = U(g)\Psi$.\footnote{Of, course, you may ask, how do we even know that we can find a unitary representation for a particular Lie group? For the unitary group this is by definition, but is it possible in general? It turns out that this is always true for {\bf compact Lie groups} which includes many of the matrix groups $U(n)$, $SU(n)$, $O(n)$, $SO(n)$, $Sp(2n)$, and $Sp(n)$.}  
With an exponential map we can write this representation as $U(g) = e^{i\lambda^a t_a}$, where $\lambda^a$ are the parameters and $t_a$ are hermitian operators, as you may perhaps have expected.

As in ordinary gauge theories, we simply copy the structure of the abelian (super)gauge transformation, and transform (a vector of) scalar superfields $\Phi$ under a non-abelian group as
\[\Phi \to \Phi'=e^{iq \Lambda^aT_a}\Phi,\]
in a non-abelian supergauge transformation. Here $q$ is the charge of $\Phi$ under $G$. At this point we need to choose a particular representation, reflected in a particular choice for the generators which we write as $T_a$. Since for gauge groups we are almost exclusively interested in groups defined by a matrix representation, $U(g)$ will be a matrix with dimension fixed by the dimension chosen for the representation. Again we can easily show that we must require that the $\Lambda^a$ has the defining property of a left-handed scalar superfield for $\Phi$ to transform to a left-handed scalar superfield.

For the superpotential to be gauge invariant we must now have:
\begin{eqnarray}
t_iU_{ir} &=& t_r ~~\text{or}~~ t_i = 0, \\
m_{ij}U_{ir} U_{js} &=& m_{rs} ~~\text{or}~~m_{ij}=0, \\
 \lambda_{ijk}U_{ir} U_{js}U_{kt}&=& \lambda_{rst} ~~\text{or}~~\lambda_{ijk}=0,
\end{eqnarray}
where the indices on $U$ are its matrix indices. 

We also want a similar construction for the kinetic terms as for abelian gauge theories, $\Phi^\dagger e^{qV^aT_a}\Phi$, to be invariant under non-abelian gauge transformations. Now, using that the generators $T_a$ are hermitian, 
\[\Phi^\dagger e^{qV^aT_a}\Phi \to \Phi'^\dagger e^{qV'^aT_a}\Phi' = \Phi^\dagger e^{-iq\Lambda^{a\dagger} T_a} e^{qV'^aT_a}e^{iq\Lambda^aT_a}\Phi,\]
so, in order to have gauge invariance, we have to require that the vector superfield $V$ transforms as:\footnote{This is independent of our choice of representation for the gauge group for the supergauge transformation.}
\begin{equation}
e^{qV'{}^aT_a} = e^{iq\Lambda^a{}^\dagger T_a}e^{qV^aT_a} e^{-iq\Lambda^a T_a}.
\end{equation}

Using the Baker-Campbell-Hausdorff formula in (\ref{eq:BKH_matrices}) we can write this as
\[V'^aT_a = V^aT_a - i(\Lambda^a-\Lambda^{a\dagger})T_a - \frac{i}{2}q[V^aT_a,(\Lambda^b + \Lambda^{b\dagger})T_b] +{\mathcal O}(\Lambda^2),\]
where the higher order terms all contain the commutator
\[ [V^aT_a,(\Lambda^b + \Lambda^{b\dagger})T_b] = V^a(\Lambda^b + \Lambda^{b\dagger}) [T_a,T_b]=  V^a(\Lambda^b + \Lambda^{b\dagger})if_{ab}^{~~c}T_c ,\]
so that the vector superfield transforms as 
\begin{equation}
V'^a = V^a - i(\Lambda^a-\Lambda^{a\dagger}) - \frac{1}{2}\,q f_{bc}^{~~a} V^b(\Lambda^{c\dagger} + \Lambda^c) +{\mathcal O}(\Lambda^2).
\label{eq:vector_superfield_nonabelian_supergauge_trans}
\end{equation}
This sensibly reduces to the definition for abelian groups in (\ref{eq:abelian_supergauge_vector_superfield}) since all the higher order terms contain the structure constant $f_{bc}^{~~a}$, which is zero for abelian groups. If we look at the component vector fields of $V^a$, $V^a_\mu$, these transform just like in a standard non-abelian  gauge theory:
\[V^a_\mu \to V'^a_\mu = V^a_\mu + i\partial_\mu(A^a - A^{a*}) - qf_{bc}^{~~a}V^b_\mu(A^c - A^{c*}),\]
in the adjoint representation of the gauge group.

The supergauge transformations of vector superfields can be written more efficiently in a representation independent way as
\[e^{V'} = e^{i\Lambda^\dagger}e^V e^{-i\Lambda},\]
where we have defined matrix superfields $\Lambda \equiv q \Lambda^aT_a$ and $V \equiv q V^aT_a$, and the inverse transformation is then given by
\[e^{-V'} = e^{i\Lambda}e^{-V} e^{-i\Lambda^\dagger},\]
such that $e^Ve^{-V} = e^{V'}e^{-V'} = 1$.\footnote{Notice that despite the non-commutative nature of the matrices involved, the identity $e^{A}e^{-A}=1$ holds. See Exercise~\ref{chap:groups}.\ref{ex:expmapprop}.}



%%%%%%%%%%%%%%%%%%%
\section{Supersymmetric field strength}
\label{sec:fieldstrength}
%%%%%%%%%%%%%%%%%%%
There is one type of term missing from the supersymmetric Lagrangian we are constructing, namely field strength terms that make the gauge fields dynamical, {\it e.g.}\ terms to describe the electromagnetic field strength term $\mathcal{L}\sim-\frac{1}{4}F^{\mu\nu}F_{\mu\nu}$ in QED.

\df{The {\bf supersymmetric field strength} for a vector superfield $V$ is defined by the spinor matrix scalar superfields $W_A$ and $\bar W_{\dot{A}}$ given by
\begin{eqnarray}
W_A  &\equiv& -\frac{1}{4}\bar{D}\bar{D}e^{-V}D_A e^V,\\
\bar{W}_{\dot{A}} &\equiv& -\frac{1}{4}{D}{D}e^{-V}\bar{D}_{\dot{A}} e^V,
\end{eqnarray}
where again $V\equiv qV^aT_a$.}
For an abelian gauge field this definition reduces to
\begin{eqnarray*}
W_A &=& -\frac{1}{4}\bar{D}\bar{D}D_A V,\\
\bar{W}_{\dot{A}} &=& -\frac{1}{4}{D}{D}\bar{D}_{\dot{A}} V,
\end{eqnarray*}
where $V$ is simply the vector superfield of the gauge group.

We can show that the components of $W_A$ ($\bar{W}_{\dot{A}}$) are left-handed (right-handed) scalar superfields, and that both ${\rm Tr}[W^AW_A]$ and ${\rm Tr}[\bar{W}_{\dot{A}}\bar{W}^{\dot{A}}]$ are supergauge invariant constructions, and thus potential terms in the supersymmetry Lagrangian. Firstly,
\[\bar{D}_{\dot{A}}W_A = -\frac{1}{4}\bar{D}_{\dot{A}}\bar{D}\bar{D}e^{-V}D_A e^V = 0,\] because from Eq.~(\ref{eq:D3}), $\bar{D}^3 = 0$, and similarly for $\bar{W}_{\dot{A}}$, showing that they are both scalar superfields of their particular type, and as such can be used to form a supersymmetric Lagrangian. 

Under a supergauge transformation we have:
\begin{eqnarray}
W_A \to W_A' &=& -\frac{1}{4}\bar{D}\bar{D}e^{i\Lambda}e^{-V}e^{-i\Lambda^\dagger}D_Ae^{i\Lambda^\dagger}e^{V}e^{-i\Lambda } \nonumber\\
 \text{ $(\bar{D}_{\dot{A}} \Lambda = 0)$\indent \indent}&=& -\frac{1}{4}e^{i\Lambda}\bar{D}\bar{D}e^{-V}e^{-i\Lambda^\dagger}D_Ae^{i\Lambda^\dagger}e^{V}e^{-i\Lambda } \nonumber\\
  \text{ $(D_{{A}} \Lambda^\dagger = 0)$\indent \indent}&=& -\frac{1}{4}e^{i\Lambda}\bar{D}\bar{D}e^{-V}D_Ae^{V}e^{-i\Lambda } \nonumber\\
&=& -\frac{1}{4}e^{i\Lambda}\bar{D}\bar{D}e^{-V}[(D_Ae^{V})e^{-i\Lambda } + e^V(D_Ae^{-i\Lambda})] \nonumber\\
&=& e^{i\Lambda}W_A e^{-i\Lambda}-\frac{1}{4}e^{i\Lambda}\bar{D}\bar{D}D_Ae^{-i\Lambda}.\label{eq:fieldstrengthtrans1}
\end{eqnarray}
We are free to add zero to (\ref{eq:fieldstrengthtrans1}) in the form of $-\frac{1}{4}e^{i\Lambda}\bar{D}D_A\bar{D}e^{-i\Lambda}=0$,\footnote{Which is zero because again $\Lambda$ is a left-handed scalar superfield, so $\bar{D}_{\dot{A}} \Lambda = 0$.} giving
\begin{eqnarray*}
W_A' &=& e^{i\Lambda}W_A e^{-i\Lambda}-\frac{1}{4}e^{i\Lambda}\bar{D}\{\bar{D},D_A\}e^{-i\Lambda}\\
&=& e^{i\Lambda}W_A e^{-i\Lambda}+\frac{1}{2}e^{i\Lambda}\bar{D}_{\dot{A}}\sigma^\mu{}_{A\dot{B}}\epsilon^{\dot{A}\dot{B}}P_\mu e^{-i\Lambda}\\
&=& e^{i\Lambda}W_A e^{-i\Lambda},
\end{eqnarray*}
where we have used Eq.~(\ref{eq:D2}) to replace the anti-commutator. This means that the following trace is gauge invariant:
\begin{eqnarray*}
\Tr[W'^AW'_A] &=& \Tr[e^{i\Lambda}W^A e^{-i\Lambda}e^{i\Lambda}W_A e^{-i\Lambda}]\\
&=& {\rm Tr}[e^{-i\Lambda}e^{i\Lambda}W^A W_A ] = {\rm Tr}[W^AW_A],
\end{eqnarray*}
and can be used in the Lagrangian.

If we expand $W_A$ in the component fields we find, as we might have hoped, that it contains the ordinary field strength tensor:
\[F_{\mu\nu}^a = \partial_\mu V^a_\nu - \partial_\nu V^{a}_{\mu} + qf_{bc}{}^aV_\mu^bV_\mu^c\]
and that the trace indeed contains terms with $F_{\mu\nu}^{a}F^{\mu\nu a}$.



%%%%%%%%%%%%%%%%%%%%%%%%%%%%%%%
\section{The (almost) complete supersymmetric Lagrangian}
%%%%%%%%%%%%%%%%%%%%%%%%%%%%%%%
We can compile all our results up to now to write down the complete Lagrangian for a supersymmetric theory with (possibly) non-abelian gauge symmetries:\footnote{There is no hermitian conjugate of the field strength term, and a slightly odd normalisation using 1/2 instead of 1/4. This is because the term can be proven to be real, although this is sometimes overlooked in the literature when authors instead use \[\mathcal{L}\sim\frac{1}{4T(R)q^2}\delta^2(\bar{\theta}){\rm Tr}[W^AW_A]+\frac{1}{4T(R)q^2}\delta^2(\theta){\rm Tr}[\bar W_{\dot A}\bar W^{\dot A}],\] which admittedly looks more symmetric with respect to the superpotential part.}
\begin{equation}
\mathcal{L}=\Phi^\dagger e^V \Phi + \delta^2(\bar{\theta}) W[\Phi] + \delta^2(\theta) W[\Phi^\dagger]+\frac{1}{2T(R)q^2}\delta^2(\bar{\theta})\Tr[W^AW_A].
\label{eq:SUSY_Lagrangian}
\end{equation}
Here, $T(R)$ given by $T(R)\delta_{ab}=\Tr[T_aT_b] $ is called the {\bf Dynkin index} of the representation $R$ of the gauge group using the generators $T_a$. This number is representation dependent, and deeply connected to the corresponding eigenvalues of the Casimir operators belonging to the irrep, but independent of which generator we choose to calculate it from.

The Dynkin index appears in the field strength term to correctly normalise the energy density for the chosen representation $R$ of the gauge group. To see that this factor cancels in a natural way, note that since the matrix structure of $W_A$ is spanned by $T_a$ for a given representation, we can write $W_A=qW_A^a T_a$, where $W_A^a$ are spinor superfields, and where we have taken out a common factor of the charge $q$ that appears in all terms of $W_A$. Then
\begin{equation}
{\rm Tr}[W^AW_A] = q^2W^{aA}W_A^b{\rm Tr}[T_aT_b] = q^2W^{Aa}W_A^b\delta_{ab}T(R) = q^2T(R)W^{aA}W_A^a.
\end{equation}

In addition to the above, it is also possible to add a pure vector superfield term as part of the Kähler potential, which is not constructed from scalar superfields, of the form $\mathcal{L}_{FI} \sim -kV$ where $V$ is the vector superfield and $k$ some constant. This kind of term is called a {\bf Fayet--Iliopoulos term}. However, this is not possible for non-abelian gauge groups since a term $-kV^a$ could not be supergauge invariant.\footnote{Why is the abelian case supergauge invariant then? Well, the superspace integration in the action will isolate the $D$-term from $V$, and this term is itself invariant under abelian supergauge transformations, see Eq.~(\ref{eq:supergaugetrans_Dterm}). For non-abelian supergauge transformations the various $V^a$ fields mix as seen in (\ref{eq:vector_superfield_nonabelian_supergauge_trans}). }



%%%%%%%%%%%%%%%%
\section{Finding the equations of motion}
\label{sec:eom}
%%%%%%%%%%%%%%%%

To find the equations of motion from our supersymmetric Lagrangian construction we can now perform the integration over superspace coordinates $\theta$  and $\bar\theta$ in the action and then apply the standard Euler--Lagrange field equations that minimise the action for the generic component fields $\phi_i$ of the Lagrangian $\mathcal L$ after the superspace integration:
\begin{equation}
\frac{\partial\mathcal{L}}{\partial\phi_i}-\partial_\mu\left(\frac{\partial\mathcal{L}}{\partial(\partial_\mu\phi_i)}\right)=0.
\end{equation}
We will now discuss some of the general properties of this solution without specifying field content or gauge groups for the Lagrangian. 

As already intimated the auxiliary component fields with unusual mass dimension, $F_i$ and $D^a$, from scalar $\Phi_i$ and vector $V^a$ superfields, respectively, will be eliminated.\footnote{In the following we use the notation in Eqs.~(\ref{eq:leftscalarsuperfield}) and (\ref{eq:vectorsuperfieldWZ}).} 
Starting with the $F_i$-fields these occur only in the kinetic terms and the superpotential. From the kinetic term $\Phi^\dagger_i e^V \Phi_i$ the contribution is $|F_i|^2$ as this is the only term with $F_i$ that can survive the $\theta$-integration if we write the vector superfields in the Wess-Zumino gauge. In the superpotential $W[\Phi_i]$ only the terms where a number of scalar fields $A_j$ with no $\theta$-coordinate multiply $F_i$ can survive the integration. The derivatives of these terms with respect to $F_i$ can be expressed in terms of the superpotential functional as 
\begin{equation}
W_i \equiv \frac{\partial W[A_1, ... , A_n]}{\partial A_i},
\end{equation}
where the scalar superfields have been replaced by their scalar component fields. This means that the Euler--Lagrange equation for $F_i^*$ is
\[\frac{\partial \mathcal{L}}{\partial F_i^*} = F_i + W_i^* = 0,\]
which is used to eliminate $F_i$. 

For a concrete example of how this can be worked out to arrive at a complete action integral for a single scalar superfield, see Exercise~\ref{chap:lagrangian}.\ref{ex:simpleL}. Generalising this to an expression for the action for any number of scalar superfields $\Phi_i$ in terms of their component fields, ignoring gauge interactions, gives:
\[S = \int d^4x\{i\partial_\mu \bar{\psi}_i\sigma^\mu \psi_i - A^*_i\Box A_i -\frac{1}{2}W_{ij}\psi_i\psi_j - \frac{1}{2}W^*_{ij}\bar{\psi}_i\bar{\psi}_j - |W_i|^2\}\]
where $W_{ij}$, given by
\begin{equation}
W_{ij} \equiv \frac{\partial^2W[A_1, ...  ,A_n]}{\partial A_i \partial A_j},
\label{eq:fermionic_mass_matrix}
\end{equation}
is called the {\bf fermionic mass matrix}.


The $D^a$-fields occur in the kinetic terms and the field strength terms.  In the kinetic term, in the Wess--Zumino gauge, it is relatively easy to see that $D^a$ appears only once in the expansion of the exponential as $qA_i^*T_{ij}^aA_jD^a$ in the action. In the field strength term one can show with some more effort that the contribution is $D^aD^a$ in the Wess--Zumino gauge. For abelian gauge fields there is also a possible $kD$ contribution from the Fayet--Iliopoulos term. This leads to the Euler--Lagrange  equation 
\[\frac{\partial \mathcal{L}}{\partial D^a} = qA_i^*T_{ij}^aA_j + 2D^a = 0,\]
which can be used to eliminate $D^a$.

Note that the terms discussed here with $F$- and $D$-fields exhaust all the non-derivative terms that only have scalar fields since no terms with only $A$-fields can survive the $\theta$-integration. Thus the complete non-derivative scalar field contribution to the Lagrangian, called the {\bf scalar potential}, can be written as
\begin{equation}
V(A_i, A_i^*) = \sum_i |F_i|^2+\sum_a D^aD^a = \sum_i\left|\frac{\partial W[A_1,... ,A_n]}{\partial A_i}\right|^2+ \frac{1}{4} \sum_a q^2 (A_i^*T_{ij}^aA_j)^2.
\label{eq:scalarpotglobal}
\end{equation}



%%%%%%%%%%%%%%%%
\section{$R$-symmetry}
\label{sec:R_symmetry}
%%%%%%%%%%%%%%%%
The Lagrangian we have constructed is invariant under both internal gauge transformations and external (coordinate) super-Poincaré transformations, where the latter has been built in through the superspace coordinates. The Minkowski part of superspace, the four-vector coordinates, transform under the Lorentz group, and since all our Lagrangian ingredients are Lorentz scalars they are invariant. We can now ask the question if there is any transformation of the superspace coordinates $\theta$ and $\bar\theta$ that the Lagrangian is also invariant under. 

In Sec.~\ref{sec:superalgebra} we claimed there there is no nontrivial interaction between a gauge group and the supersymmetry generators for $N=1$ supersymmetry. This is not entirely true as we may have non-trivial commutators for the generator $R$ of an abelian group, {\it i.e.}
\begin{equation}
[Q_A, R] = Q_A, \quad  [\bar Q_{\dot A}, R ] = -\bar Q_{\dot A}.
\end{equation}
Using this we define a $U(1)_R$ transformation called {\bf $\mathbf R$-symmetry} as
\begin{equation}
\theta_A \to \theta_A' = e^{i\alpha} \theta_A, \quad \bar\theta_{\dot A} \to  \bar\theta'_{\dot A}= e^{-i\alpha} \bar\theta_{\dot A},
\end{equation}
where $\alpha$ is the parameter of the transformation and the charge of $\theta$ and $\bar\theta$ under the transformation is 1 and $-1$, respectively.
If a Lagrangian is to be invariant under such a transformation then a superfield $\Phi$ must transform as
\begin{equation}
\Phi(x,\theta,\bar\theta)\to \Phi'(x,\theta,\bar\theta)= e^{ir_\Phi\alpha}\Phi(x,e^{-i\alpha} \theta,e^{i\alpha} \bar\theta),
\end{equation}
where $r_\Phi$ is the charge of that superfield under the transformation. The charge of a product of two superfields is then just the sum of the charges of the fields. This is required so that the kinetic term in the Lagrangian is invariant under the transformation.

This means that for a scalar (left- or right-handed) superfield the component fields $A$, $\psi$, and $F$ must transform as
\begin{equation}
A\to e^{ir_\Phi\alpha}A,\quad \psi\to e^{i(r_\Phi-1)\alpha} \psi,\quad F\to e^{i(r_\Phi-2)\alpha} F,
\end{equation}
while vector superfields must have zero $R$-charge since they are real, and their component fields, in the Wess-Zumino gauge must thus transform as
\begin{equation}
V_\mu \to V_\mu ,\quad \lambda \to e^{i\alpha} \lambda,\quad D\to D.
\end{equation}
We can check that all the terms in the Lagrangian (\ref{eq:SUSY_Lagrangian}) are invariant under this transformation, with the exception of the superpotential $W$, which must have total $R$-charge 2. This will exclude most allowed charge assignments for the the scalar superfields, and in the minimal supersymmetruc models we shall look at there will be no viable continuous $R$-symmetry. However, $R$-symmetry has importance in building models of supersymmetry breaking, and could remain as a broken symmetry in the model.



%%%%%%%%%%%%%%%%
\section{The hierarchy problem}
%%%%%%%%%%%%%%%%
There is a fundamental problem with having scalar particles in a quantum field theory. Let us take the Standard Model Higgs boson $h$ as an example, however, the following would be true for any scalar particle. At tree level its behaviour is controlled by the Standard Model scalar potential
\begin{equation}
V(h)=-\mu^2|h|^2+\lambda |h|^4,
\label{eq:SM_Higgs_potential}
\end{equation}
where $\mu$ is the tree-level mass. and $\lambda$ is its self-coupling. If we naively calculate loop-corrections to its mass in self-energy diagrams like the ones shown in Fig.~\ref{fig:hierarchy}, where $f$ is a fermion and $s$ some other scalar, they both diverge due to their loop momenta integrals, meaning they are infinite. This then needs what is called {\bf regularisation} in order to yield a finite answer. 

\begin{figure}
\centering
\begin{tikzpicture}
\begin{feynman}
%
\diagram [layered layout, horizontal=b to c] {
a [particle=$h$] -- [scalar] b [dot, black, label=$\lambda_f\hphantom{m}$]
-- [fermion, half left, out=90, in=90, edge label = $f$] c [dot, black, label=$\hphantom{nm}\lambda_f$]
-- [fermion, half left, out=90, in=90, edge label = $\bar{f}$] b,
c -- [scalar] d [particle=$h$],
};
%
\diagram [xshift=6cm, yshift=-0.6cm, layered layout, horizontal=b to c] {
a [particle=$h$] -- [scalar] b [dot, black,  label=$\lambda_s$]
-- [scalar] c [particle=$h$],
};
\path (b)--++(90:0.8) coordinate (A);
\path (b)--++(90:1.6) coordinate (B);
\draw [dashed] (A) circle(0.8);
\draw (B) node[above] {$s$};
%
%\diagram [xshift=10cm, layered layout, horizontal=b to c] {
%a [particle=$h$] -- [scalar] b [dot, black, label=$\lambda_V\hphantom{m}$]
%-- [boson, half left, out=90, in=90, edge label = $V$] c [dot, black, label=$\hphantom{nm}\lambda_V$]
%-- [boson, half left, out=90, in=90, edge label = $\bar{V}$] b,
%c -- [scalar] d [particle=$h$],
%};
%
\end{feynman}
\end{tikzpicture}
\caption{Higgs self-energy diagrams with fermions $f$ (left) and scalars $s$ (right). The dots indicate vertices with the given coupling strength.}  
\label{fig:hierarchy}
\end{figure}

There are different ways of achieving regularisation. Since we know that the Standard Model is an incomplete theory, at least when we go up to Planck scale energies where we need an unknown quantum theory of gravity, we can introduce a {\bf cut-off regularisation} where we limit the integral in the loop-correction to momenta below a scale $\Lambda_{UV}$. Then the loop-correction to the Higgs mass is, at leading order in $\Lambda_{UV}$,
\begin{equation}
\Delta m_h^2 = -\frac{|\lambda_f|^2}{8\pi^2}\Lambda_{UV}^2 + \frac{\lambda_s}{16\pi^2}\Lambda_{UV}^2+\ldots \label{eq:higgsqdiv}
\end{equation}
where $\lambda_f$ and $\lambda_s$ are the couplings of $f$ and $s$ to the Higgs, respectively, from the Lagrangian interaction terms $\mathcal L \sim \lambda_f \bar\psi\psi h+\lambda_s |s|^2 |h|^2$. The dots represents terms that are less than quadratically divergent in terms of the cut-off scale, the first missing terms being logarithmically divergent.
Picking $\Lambda_{UV}$  suggestively as the Planck scale, $\Lambda_{UV} =M_P = 2.44\times 10^{18}$ GeV, meaning that the quantum corrections to the Higgs mass are some 15 orders of magnitude greater than its measured value. 

We observe that the difference in sign between the fermions and bosons in (\ref{eq:higgsqdiv}) means that it could {\it in principle} be possible that these huge contributions cancel to keep $m_h \sim 125$\,GeV as measured,\footnote{Could we not fix this by introducing a very large Lagrangian Higgs mass $\mu$? Not really, in that the tree-level Higgs mass is constrained by the properties of the other electroweak masses, {\it i.e.}\ the $Z$ and $W$-bosons.} however, as physicists we should worry why the Universe seems to be organised in such a strange way without any specific cause. This is known as the {\bf Higgs fine tuning problem}, or the scale {\bf hierarchy problem} because it is fundamentally a problem of why the electroweak scale at around 100 GeV where the Higgs lives, is so much lighter than the Planck scale of $10^{18}$ GeV, when the former should in principle be pulled up to the latter by the sensitivity of scalars to loop corrections.\footnote{With some background in renormalisation you may ask: What about choosing dimensional regularisation instead where there is no cut-off scale? That would in principle work, however, as soon as you introduce {\it any} new particle (significantly) heavier than the Higgs there is still a quadratic correction of the form of (\ref{eq:higgsqdiv}) with the new particle masses as the scale, meaning that we cannot complete the Standard Model at a higher scale without reintroducing the problem.}

Enter supersymmetry to the rescue: with unbroken supersymmetry we find that we automatically have $|\lambda_f|^2 = \lambda_s$ and exactly twice as many scalar as fermion degrees of freedom running around in loops. This provides a magical exact cancellation of the quadratic divergence in Eq.~(\ref{eq:higgsqdiv}). To see that this relation between the couplings holds, notice that it is the superpotential term $W \sim \lambda_{ijk}\Phi_i\Phi_j \Phi_k$ which is responsible for generating Lagrangian Yukawa terms of the form $\lambda_{ijk} \psi_i\psi_j A_k$ that couple two fermions to a scalar $A_k$. If the superfield $\Phi_k$ is the one that contains the Higgs boson this means that $A_k$ is the Higgs boson field $h$, and it couples to the right-handed Weyl-spinors $\psi_i$ and $\psi_j$  in the superfields $\Phi_i$ and $\Phi_j$  as $\lambda_{ijk}\psi_i\psi_j h$, with the coupling $\lambda_{ijk}$. From the superpotential with right-handed scalar superfields the same term appears with right-handed Weyl-spinors $\lambda_{ijk}\bar\psi_i\bar\psi_j h^*$. These two terms can be combined to the same interaction term with a single Dirac fermion $\psi$,  $\lambda_{ijk}\bar\psi\psi h$, as in the Standard Model, giving $\lambda_f = \lambda_{ijk}$. There are no other such terms coupling two fermions to a scalar in the whole Lagrangian.

At the same time, in the scalar potential (\ref{eq:scalarpotglobal}), that same superpotential term is responsible for the terms
\begin{equation}
V(A,A^*) \sim \left|\frac{\partial W}{\partial A_i}\right|^2 = |\lambda_{ijk}|^2A^*_jA^*_kA_jA_k.
\end{equation}
If $A_j$ is the second scalar $s$ in Fig.~\ref{fig:hierarchy}, then this term becomes $|\lambda_{ijk}|^2|s|^2|h|^2$ and we have $\lambda_s=|\lambda_{ijk}|^2$. As a result $\lambda_f^2=\lambda_s$.

The fact that there are loop contributions from two scalars for each fermion due to the same number of states (two scalar states per fermion state), means that unbroken supersymmetry predicts an exact cancellation in Eq.~(\ref{eq:higgsqdiv}).\footnote{To keep the argument simple we have avoided the contributions from vector bosons, however, we can show that these also cancel exactly against the contributions from their fermionic partners.} Notice that there is nothing here that is special about the Higgs boson, this mechanism will in fact protect all scalar particles from quadratic corrections to their mass in supersymmetric models.



%%%%%%%%%%%%%%%%%%%%%%
%\section{The non-renormalization theorem}
%\label{sec:non-renorm}
%%%%%%%%%%%%%%%%%%%%%%
%With our generic supersymmetric Lagrangian in Eq.~(\ref{eq:SUSY_Lagrangian}) we should really ask ourselves whether we can regularize the theory, {\it i.e.}\ is there a finite number of renormalisation constants/counter terms to make all measurable predictions finite? And if so, what are they?
%
%You may not be so surprised that the answer is yes, and indeed we have already used one of the restrictions this gives on the possible terms in our superpotential construction. Furthermore, we can prove the following theorem with a funny name\ldots
%
%\theo{{\bf Non-renormalisation theorem} (Grisaru, Roach and Siegel, 1979~\cite{Grisaru:1979wc})\\
%All higher order contributions to the effective supersymmetric action $S_{\rm eff}$ can be written:
%\begin{equation}
%S_{\rm eff} = \sum_n \int d^4x_i...d^4x_nd^4\theta \,F_1(x_1,\bar{\theta},\theta)\times...\times F_n(x_1,\bar{\theta},\theta)\times G(x_1, ..., x_n),
%\end{equation}
%where $F_i$ are products of the external superfields and their covariant derivatives, and $G$ is a supersymmetry invariant function.}
%
%So, why is the name funny? Well, mainly because it is not about not being able to renormalize the theory, but about about not {\it needing} to renormalize certain parts of it. The theorem has two important consequences:\footnote{The theorem is for unbroken supersymmetry.}
%\begin{enumerate}
%\item The couplings of the superpotential do not need separate normalization.
%\item There is zero vacuum energy in global unbroken SUSY. In other words, $\Lambda = 0$ in general relativity.
%\item Quantum corrections cannot (perturbatively) break supersymmetry.
%\end{enumerate}
%
%Let us try to argue how these consequences come about. From the non-renormalization theorem we know that there are no counter terms needed for superpotential terms, because superpotential terms have lower $\theta$ integration than found in all the possible higher order contributions in the non-renormalisation theorem. This means that we can relate the bare fields $\Phi_0$ and couplings $g_0$, $m_0$ and $\lambda_0$ to the renormalized fields $\Phi$ and couplings $g$, $m$ and $\lambda$, by
%\begin{eqnarray}
%g_0\Phi_0 &=& g\Phi, \\
%m_0\Phi_0\Phi_0&=& m\Phi\Phi, \\
%\lambda_0\Phi_0\Phi_0\Phi_0&=&\lambda\Phi\Phi\Phi.
%\end{eqnarray}
%If we let scalar superfields be renormalized by the {\bf counterterm} $Z$, $\Phi_0 = Z^{1/2}\Phi$, vector superfields by $Z_V$, $V_0 = Z_V^{1/2}V$, coupling constant $g$ by $Z_g$, $g_0 = Z_g g$, $m$ by $Z_m$, $m_0 = Z_m m$, and $\lambda$ by $Z_\lambda$, $\lambda_0 = Z_\lambda \lambda$, then
%\begin{eqnarray}
%Z_gZ^{1/2} &=& 1 \\
%Z_mZ^{1/2}Z^{1/2} &=& 1 \\
%Z_\lambda Z^{1/2}Z^{1/2}Z^{1/2}&=& 1
%\end{eqnarray}
%This set of equations can be solved for $Z_g$, $Z_m$ and $Z_\lambda$ in terms of $Z^{1/2}$ so no separate renormalization except for the superfields $\Phi$ and $V$ is needed.
%
%The second consequence comes about because vaccum diagrams have no external fields. This means that the integration $\int d^4\theta$ in $S_{\rm eff}$ gives zero for the contribution from these diagrams. The same argument leads to $V(A,A^*) = 0$ after quantum corrections.
%
%In practice the regularisation of supersymmetric models is tricky. Using so-called DREG (dimensional regularisation) with modified minimal subtraction ($\overline{MS}$) fails because working in $d=4-\epsilon$ dimensions violates the supersymmetry in the Lagrangian. In practice DRED (dimensional reduction) with $\overline{DR}$ is used, where all the algebra is done in four dimensions, but integrals are done in $d=4-\epsilon$ dimensions. However, this leads to its own problems with potential ambiguities in higher loops.



%%%%%%%%%%%%%
\section{Vacuum energy}
\label{sec:vacuum_energy}
%%%%%%%%%%%%%
To explain the measured accelerated expansion of the universe one can introduce a constant term $\Lambda$ in Einstein's field equation, 
\begin{equation}
R_{\mu\nu}-\half R g_{\mu\nu}-g_{\mu\nu}\Lambda = \frac{1}{M_P^2}T_{\mu\nu}
\end{equation}
and this contribution has been coined {\bf dark energy}. 
The measured value of dark energy in terms of an energy scale is $\Lambda_{DE} \simeq 2.4\times 10^{-3}$\,eV. To get some sense of this energy scale the rest mass of the lightest charged particle, the electron, is around 0.511 MeV, while the energy of a photon in visible light is of the order of 1 eV.

This energy can be interpreted as vacuum energy, {\it i.e.}\ the energy predicted from the contribution of Feynman diagrams with {\it no} external particles. However, the predicted value of the vacuum energy in the Standard Model is $\Lambda_{VE} \sim M_P$.\footnote{The origin of this is just the same as the quadratic divergence for the Higgs mass. It is the same type of diagrams contributing, only without external legs.} If we want to compare energy densities, we should compare $\Lambda_{DE}^4$ to $\Lambda_{VE}^4$,\footnote{In natural units the unit of energy density is $[\rho]=[E/V]=M/L^3=M^4$.} which means that the prediction is off from the measurement by some 120 orders of magnitude, which is said to be a record for the greatest ever discrepancy between theory and experiment. This problem is the {\bf hierarchy problem for vacuum energy}.

Can supersymmetry save us here as well? Sort of. For an unbroken global supersymmetry we can use the supersymmetry {\bf non-renormalisation theorem}\footnote{The name of this theorem is a bit funny. Why? Well, mainly because the result is not about not being able to renormalise the theory, but about about not {\it needing} to renormalise certain parts of it. Another consequence of the theorem, which gives it its name, is that the couplings of the superpotential do not need separate renormalisation. The renormalisation of the superfields suffices.}
of Grisaru, Roach and Siegel (1979)~\cite{Grisaru:1979wc} to show that the prediction in supersymmetry is exactly $\Lambda_{VE} = 0$.

\theo{{\bf Non-renormalisation theorem}\\
All higher order contributions to the effective supersymmetric action $S_{\rm eff}$  for a process can be written:
\begin{equation}
S_{\rm eff} = \sum_n \int d^4x_i...d^4x_nd^4\theta \,F_1(x_1,\theta,\bar{\theta})\times...\times F_n(x_1,\theta,\bar{\theta})\times G(x_1, ..., x_n),
\end{equation}
where $F_i$ are products of the $n$ external superfields in the process and their covariant derivatives, and $G$ is a supersymmetry invariant function.}

While this is a rather technical statement, that fact that vacuum diagrams do not have external superfields means that the higher order contributions are all zero, and since there are no tree-level contributions for vacuum diagrams -- you must have loops to not have external legs -- there can be no contribution at all. 

This was a victory for supersymmetry before the discovery of dark energy, when the problem was instead to prove that $\Lambda=0$, modulo the fact that the breaking of supersymmetry, as we will be doing in Chapter \ref{chap:breaking}, changes this prediction. As we shall see there, the scale of the contribution to the vacuum energy in broken supersymmetry has to be the mass scale of the supersymmetric particles, so with for example $m_{SUSY} \simeq 2$\,TeV as this scale, we have $m_{SUSY}/\Lambda_{DE} \simeq 10^{15}$, some 15 orders of magnitude too large, which is better than the Standard Model prediction of $M_P/\Lambda_{DE} = 10^{30}$, but still a bit off the measured value.\footnote{So, one would be tempted to say that supersymmetry has solved half the problem. On a more serious note, there is a significant difference in that the Standard Model prediction requires a cut-off to be finite, while supersymmetry predicts a finite but too large value.}


So now we are left with showing that the contribution is very small but non-zero, which is in general thought to be a much harder problem than finding models where it  is exactly zero.
However, in supergravity something interesting happens. Introducing a local supersymmetry the scalar potential is not simply given by the superpotential derivatives in (\ref{eq:scalarpotglobal}), but instead is (ignoring the effects of gauge fields)
\begin{equation}
V(A,A^*) = e^{K/M_P^2}\left[K^{-1}_{ij}(D_iW)(D_jW^*)-\frac{3}{M_P^2}|W|^2\right],
\end{equation}
where $K_{ij}=\partial_i\partial_jK[A,A^*]$ is the {\bf Kähler metric},  $D_i$ the {\bf Kähler derivative} $D_i = \partial_i +\frac{1}{M_P^2}(\partial_iK)$, and all the derivatives are with respect to the scalar fields in the Kähler potential $K$ and superpotential $W$. In the $M_P\to \infty$ limit, the low energy limit, we see that we recover the flat space result of Eq.~(\ref{eq:scalarpotglobal}). What is important to notice is that  there is now a second {\it negative} term in the potential that can in principle cancel the positive supersymetry breaking contribution, however, given the large size of the breaking contribution this will come at the price of fantastic fine-tuning unless some mechanism can be found where this is natural.



%%%%%%%%%%%%%
\section{Excercises}
%%%%%%%%%%%%%

\begin{Exercise}[]
\label{ex:simpleL}
Write down the Lagrangian and find the action of the simplest possible supersymmetric field theory with a single scalar superfield, without gauge transformations, in terms of component fields, and show that it contains no kinetic terms for the $F$-field. Then show how the $F$-field can be eliminated by the equations of motion. {\it Hint:} The kinetic part of the action will turn out to be
\begin{equation}
S_\text{kin}=\int d^4x \left\{-A^*(x)\square A(x)+|F(x)|^2+i\partial_\mu\psi(x)\bar\sigma^\mu\psi(x)\right\}.
\end{equation}
To show this you may have use of the identities in Sec.~\ref{sec:Weylspinor_calc}.
\end{Exercise}

\begin{Answer}
The Lagrangian for a single scalar superfield $\Phi$ is
\begin{equation}
\mathcal{L}=\Phi^\dagger \Phi + \delta^2(\bar{\theta}) (g\Phi+m\Phi\Phi+\lambda\Phi\Phi\Phi) + \delta^2(\theta) (g\Phi^\dagger+m\Phi^\dagger\Phi^\dagger+\lambda\Phi^\dagger\Phi^\dagger\Phi^\dagger). 
\end{equation}

Using the following expressions for the scalar superfield in terms on component fields
\begin{eqnarray}
\Phi(x, \theta, \bar{\theta}) &=& A(x) + i(\theta\sigma^\mu \bar{\theta})\partial_\mu A(x) - \frac{1}{4}\theta\theta\bar{\theta}\bar{\theta}\Box A(x) + \sqrt{2}\theta \psi(x) - \frac{i}{\sqrt{2}}\theta\theta\partial_\mu \psi(x)\sigma^\mu\bar{\theta} \nonumber\\
&&+ \theta\theta F(x).\nonumber\\
\Phi^\dagger (x, \theta, \bar{\theta}) &=& A^*(x) - i(\theta\sigma^\mu \bar{\theta})\partial_\mu A^*(x) - \frac{1}{4}\theta\theta\bar{\theta}\bar{\theta}\Box A^*(x) + \sqrt{2}\bar{\theta}\bar{\psi}(x) + \frac{i}{\sqrt{2}}\bar{\theta}\bar{\theta}\theta \sigma^\mu\partial_\mu \bar{\psi}(x)\nonumber\\
&& +\bar{\theta}\bar{\theta} F^*(x).\nonumber
\end{eqnarray}
we have for the kinetic term,
\begin{eqnarray}
\int d^4\theta \,\Phi^\dagger\Phi&=& - \frac{1}{4}A\Box A^* -\frac{1}{4} A^*\Box A  +  |F|^2 \nonumber \\
&&+\int d^4\theta \left\{ (\theta\sigma^\mu \bar{\theta})\partial_\mu A^*(\theta\sigma^\nu \bar{\theta})\partial_\nu A
+i\bar{\theta}\bar{\theta}\theta \sigma^\mu\partial_\mu \bar{\psi}\theta \psi - i\bar{\theta}r{\psi}\theta\theta\partial_\mu \psi\sigma^\mu\bar{\theta}\right\},\nonumber
\end{eqnarray}
where we have removed the factor $\theta\theta\bar{\theta}\bar{\theta}$ by integration. Using the identities
\begin{eqnarray}
(\theta\sigma^\mu \bar{\theta})(\theta\sigma^\nu \bar{\theta})&=&\frac{1}{2}g^{\mu\nu}\theta\theta\bar{\theta}\bar{\theta},\\
\theta\sigma^\mu\partial_\mu \bar\psi \theta\psi &=& -\frac{1}{2}\psi\sigma^\mu\partial_\mu \bar\psi \theta\theta,\\
\partial_\mu\psi\sigma^\mu \bar\theta \bar\theta\bar\psi &=& -\frac{1}{2}\partial_\mu\psi\sigma^\mu \bar\psi \bar\theta\bar\theta,
\end{eqnarray}
gives
\begin{eqnarray}
\int d^4\theta \,\Phi^\dagger\Phi&=& - \frac{1}{4}A\Box A^* -\frac{1}{4} A^*\Box A +\frac{1}{2}\partial^\mu A\partial_\mu A^* +  |F|^2 \nonumber \\
&&-\frac{i}{2}\psi\sigma^\mu\partial_\mu \bar\psi+ \frac{i}{2}\partial_\mu\psi\sigma^\mu \bar\psi.\nonumber
\end{eqnarray}
Now, since
\begin{equation}
\frac{1}{2}\partial^\mu A\partial_\mu A^* = \frac{1}{2}\partial^\mu (A\partial_\mu A^*)-\frac{1}{2}A\Box A^*,
\end{equation}
and
\begin{eqnarray}
-\frac{1}{4}A\Box A^* &=& -\frac{1}{4}\partial_\mu (A\partial^\mu A^*)+\frac{1}{4}\partial_\mu A\partial^\mu A^*\nonumber\\
&=& -\frac{1}{4}\partial_\mu (A\partial^\mu A^*)+\frac{1}{4}\partial^\mu ((\partial_\mu A) A^*)-\frac{1}{4} A^*\Box A,\nonumber
\end{eqnarray}
we can write
\begin{eqnarray}
- \frac{1}{4}A\Box A^* -\frac{1}{4} A^*\Box A +\frac{1}{2}\partial^\mu A\partial_\mu A^* = -A^*\square A + {\rm total~derivatives.}
\end{eqnarray}
Using (\ref{eq:Weylspinor_etasigmapsi}) we can similarly write
\begin{eqnarray}
-\frac{i}{2}\psi\sigma^\mu\partial_\mu \bar\psi+ \frac{i}{2}\partial_\mu\psi\sigma^\mu \bar\psi
&=&\frac{i}{2}\partial_\mu \bar\psi\bar\sigma^\mu\psi- \frac{i}{2}\bar\psi\bar\sigma^\mu\partial_\mu\psi\nonumber\\
&=&i\partial_\mu \bar\psi\bar\sigma^\mu\psi- \frac{i}{2}\partial_\mu(\bar\psi\bar\sigma^\mu\psi).
\end{eqnarray}
Removing the terms with total derivatives the kinetic term is
\begin{equation}
\int d^4\theta \,\Phi^\dagger\Phi=\int d^4x \left\{-A^*(x)\square A(x)+|F(x)|^2+i\partial_\mu\psi(x)\bar\sigma^\mu\psi(x)\right\}.
\end{equation}

The (left-handed) superpotential terms are
\begin{eqnarray*}
\int d^4\theta\, \bar\theta\bar\theta\Phi &=&  F, \\
\int d^4\theta\, \bar\theta\bar\theta\Phi\Phi &=& 2AF+\psi\psi, \\
\int d^4\theta\, \bar\theta\bar\theta\Phi\Phi\Phi &=& 3A^2F+3A\psi\psi.\\
%(A(x)+ \sqrt{2}\theta \psi(x)+ \theta\theta F(x) )( A(x)  + \sqrt{2}\theta \psi(x)+ \theta\theta F(x) ) 
%&=& 2\theta\theta A(x)F(x)+\theta \theta \psi(x)\psi(x) \\
%(A(x)+ \sqrt{2}\theta \psi(x)+ \theta\theta F(x) )( A(x)  + \sqrt{2}\theta \psi(x)+ \theta\theta F(x) )( A(x)  + \sqrt{2}\theta \psi(x)+ \theta\theta F(x) ) 
%&=& 3\theta\theta  A(x)A(x)F(x) + 3A(x)\theta\theta  \psi(x)\psi(x)
\end{eqnarray*}
where we have used (\ref{eq:Weylspinor_etapsietapsi}) to rewrite the terms with $\psi$. The total action is then 
\begin{eqnarray}
S&=&\int d^4x \left\{-A^*(x)\square A(x)+|F(x)|^2+i\partial_\mu\bar\psi(x)\bar\sigma^\mu\psi(x) \right. \nonumber \\ 
&+&\left. gF(x)+ 2mA(x)F(x)+m\psi(x)\psi(x)+3\lambda A(x)^2F(x)+3\lambda A(x)\psi(x)\psi(x)+\text{h.c.} \right\}
\end{eqnarray}

The equation of motion for $F$ is then
\begin{equation}
\frac{\partial\mathcal{L}}{\partial F^*}=F+g+2mA^*+3\lambda A^{*2}=0,
\end{equation}
which we can solve for $F$. With $\frac{\partial W}{\partial A}=g+2mA+3\lambda A^2$, $F=-\frac{\partial W^*}{\partial A^*}$ and the action is
\begin{eqnarray}
S&=&\int d^4x \left\{-A^*(x)\square A(x)+i\partial_\mu\psi(x)\bar\sigma^\mu\psi(x)+m\psi(x)\psi(x)+3\lambda A(x)\psi(x)\psi(x) \right. \nonumber \\ 
&&\left.  +m\bar\psi(x)\bar\psi(x)+3\lambda A^*(x)\bar\psi(x)\bar\psi(x) -\left|\frac{\partial W}{\partial A}\right|^2\right\}.
\end{eqnarray}
Using the fermionic mass matrices in (\ref{eq:fermionic_mass_matrix}) this can be written as
\begin{eqnarray}
S&=&\int d^4x \left\{-A^*(x)\square A(x)+i\partial_\mu\bar\psi(x)\bar\sigma^\mu\psi(x)+\half\frac{\partial^2 W}{\partial A^2}\psi(x)\psi(x) \right. \nonumber \\ 
&&\left.  +\half\frac{\partial^2 W^*}{\partial A^{*2}}\bar\psi(x)\bar\psi(x)-\left|\frac{\partial W}{\partial A}\right|^2\right\}.
\end{eqnarray}

\end{Answer}


%%%
\begin{Exercise}[]
\label{ex:abelian_field_strength}
Show that the supersymmetric field strength term for an abelian gauge field can be written in terms of component fields as
\begin{eqnarray}
\mathcal L = -i\lambda(x)\sigma^\mu \partial_\mu \bar\lambda(x)+2D^2(x)-\frac{1}{4}F_{\mu\nu}(x)F^{\mu\nu}(x) -\frac{i}{4}F_{\mu\nu}(x)\tilde F^{\mu\nu}(x).
\end{eqnarray}
Try to find an argument why the last term with the dual field tensor $\tilde F^{\mu\nu}(x)$ can be ignored when finding the equations of motion.  {\it Hints:} To get started it may be productive to consider the coordinate change $y^\mu=x^\mu + i\theta\sigma^\mu\bar\theta$. You will also likely need the following algebraic relationship
\begin{equation}
(\sigma^{\mu\nu}\theta)^A F_{\mu\nu} (\sigma^{\rho\sigma}\theta)_A F_{\rho\sigma}  = -\frac{1}{2}\theta\theta [F_{\mu\nu}F^{\mu\nu} + iF_{\mu\nu}\tilde F^{\mu\nu}].
\end{equation}
\end{Exercise}

\begin{Answer}
To calculate the field strength in Wess--Zumino gauge we start from the vector superfield itself
\begin{equation}
V_{WZ} (x, \theta, \bar{\theta}) = \theta\sigma^\mu \bar{\theta}V_\mu(x) + \theta\theta\bar{\theta}\bar{\lambda}(x) + \bar{\theta}\bar{\theta}\theta\lambda (x) + \theta\theta\bar{\theta}\bar{\theta}D(x),
\end{equation}
and make the coordinate change $y^\mu=x^\mu + i\theta\sigma^\mu\bar\theta$. This gives
\begin{eqnarray}
V_{WZ} (x, \theta, \bar{\theta}) &=& \theta\sigma^\mu \bar{\theta}V_\mu(y)- i\theta\sigma^\nu\bar\theta\partial_\nu \theta\sigma^\mu \bar{\theta} V_\mu(y)  + \theta\theta\bar{\theta}\bar{\lambda}(y) + \bar{\theta}\bar{\theta}\theta\lambda (y) + \theta\theta\bar{\theta}\bar{\theta}D(y) \nonumber\\
&=& \theta\sigma^\mu \bar{\theta}V_\mu(y) + \theta\theta\bar{\theta}\bar{\lambda}(y) + \bar{\theta}\bar{\theta}\theta\lambda (y) + \theta\theta\bar\theta\bar\theta[D(y)- \frac{i}{2}\partial^\mu V_\mu(y)],
\end{eqnarray}
where we have used (\ref{eq:Weylspinor_etasigmamuetaetasigmanueta}).

Since, in this coordinate system, $D_A=\partial_A+2i(\sigma^\mu\bar\theta)_A\partial_\mu$ and $\bar D_{\dot A}=-\partial_{\dot A}$, so that $\bar D \bar D=\partial_{\dot A}\partial^{\dot A}$, we can then find the field strength as
\begin{eqnarray}
W_A&=& -\frac{1}{4} \bar D \bar D D_A V_{WZ} (y, \theta, \bar{\theta}) \nonumber\\
&=& -\frac{1}{4}\partial_{\dot A}\partial^{\dot A} \left[
(\sigma^\mu\bar\theta)_AV_\mu(y)+ i\theta_A\bar\theta\bar\theta\partial^\mu V_\mu(y)  + 2\theta_A\bar\theta\bar\lambda(y) + \bar\theta\bar\theta\lambda_A(y) + 2\theta_A\bar\theta\bar\theta D(y) \right. \nonumber\\
&& \left. +2i(\sigma^\nu\bar\theta)_A\partial_\nu \left( \theta\sigma^\mu \bar\theta V_\mu(y) + \theta\theta\bar\theta\bar\lambda(y) \right)
\right] \nonumber\\
&=& -\frac{1}{4} \partial_{\dot A}\partial^{\dot A} \left[
(\sigma^\mu\bar\theta)_AV_\mu(y)+ i\theta_A\bar\theta\bar\theta\partial^\mu V_\mu(y)  + 2\theta_A\bar\theta\bar\lambda(y) + \bar\theta\bar\theta\lambda_A(y) + 2\theta_A\bar\theta\bar\theta D(y) \right. \nonumber\\
&& \left. +2i(\sigma^\nu\bar\theta)_A\theta\sigma^\mu \bar\theta \partial_\nu V_\mu(y) - i\theta\theta\bar\theta\bar\theta(\sigma^\nu\partial_\nu\bar\lambda(y))_A \right] \nonumber\\
&=& i\theta_A\partial^\mu V_\mu(y) +\lambda_A(y) + 2\theta_A D(y) \nonumber\\
&& +2i\partial_{\dot A}\partial^{\dot A}(\sigma^\nu\bar\theta)_A\theta\sigma^\mu \bar\theta \partial_\nu V_\mu(y) - i\theta\theta(\sigma^\nu\partial_\nu\bar\lambda(y))_A \nonumber\\
&=&\lambda_A(y) + 2\theta_A D(y)+(\sigma^{\mu\nu}\theta)_A F_{\mu\nu}(y) - i\theta\theta   (\sigma^\mu \partial_\mu \bar\lambda(y))_A
\end{eqnarray}
where we have used $\partial_{\dot A}\partial^{\dot A}(\bar\theta\bar\theta)=4$ and $2i(\sigma^\nu\bar\theta)_A\theta\theta\bar\theta\partial_\nu\bar\lambda(y)=-i\theta\theta\bar\theta\bar\theta(\sigma^\nu\partial_\nu\bar\lambda(y))_A$, and where $F_{\mu\nu}=\partial_\mu V_\nu-\partial_\nu V_\mu$ is the field strength for the field $V_\mu$.

Then the terms with a $\theta\theta$ factor in $W^AW_A$ are
\begin{eqnarray}
W^AW_A|_{\theta\theta} &=&-i\theta\theta\lambda(y)\sigma^\mu \partial_\mu \bar\lambda(y)-i\theta\theta(\sigma^\mu \partial_\mu \bar\lambda(y))^A\lambda_A(y)+4\theta\theta D^2(y) \nonumber\\
&& +2D(y)\theta\sigma^{\mu\nu}\theta F_{\mu\nu}(y) +2(\sigma^{\mu\nu}\theta)^A F_{\mu\nu}(y) \theta_A D(y)+(\sigma^{\mu\nu}\theta)^A F_{\mu\nu}(y) (\sigma^{\rho\sigma}\theta)_A F_{\rho\sigma}(y) \nonumber\\
&=&-2i\theta\theta\lambda(y)\sigma^\mu \partial_\mu \bar\lambda(y)+4\theta\theta D^2(y) \nonumber\\
&& +4\theta\sigma^{\mu\nu}\theta D(y)F_{\mu\nu}(y) +(\sigma^{\mu\nu}\theta)^A F_{\mu\nu}(y) (\sigma^{\rho\sigma}\theta)_A F_{\rho\sigma}(y) \nonumber\\
&=&-2i\theta\theta\lambda(y)\sigma^\mu \partial_\mu \bar\lambda(y)+4\theta\theta D^2(y) -\theta\theta\frac{1}{2}F_{\mu\nu}(y)F^{\mu\nu}(y) -\theta\theta\frac{i}{2}F_{\mu\nu}(y)\tilde F^{\mu\nu}(y)
\end{eqnarray}
where we have used that from (\ref{eq:Weylspinor_etasigmamunupsi}) $\theta\sigma^{\mu\nu}\theta=0$ to remove a term, and rewritten
\begin{equation}
(\sigma^{\mu\nu}\theta)^A F_{\mu\nu}(y) (\sigma^{\rho\sigma}\theta)_A F_{\rho\sigma}(y)  = -\theta\theta\frac{1}{2}F_{\mu\nu}(y)F^{\mu\nu}(y) -\theta\theta\frac{i}{2}F_{\mu\nu}(y)\tilde F^{\mu\nu}(y),
\end{equation}
where $\tilde F^{\mu\nu}=\half \epsilon^{\mu\nu\rho\sigma}F_{\rho\sigma}$ is the dual field strength tensor.  

Since these terms are invariant under the change of coordinates $y^\mu=x^\mu + i\theta\sigma^\mu\bar\theta$ -- because all terms beyond the first order disappear in the expansion due to too many $\theta$-factors -- and since no other terms can survive the $\int d^4\theta$-integration in the Lagrangian, we are left with
\begin{eqnarray}
\int d^4\theta\, \bar\theta\bar\theta W^AW_A = -2i\lambda(x)\sigma^\mu \partial_\mu \bar\lambda(x)+4D^2(x)-\frac{1}{2}F_{\mu\nu}(x)F^{\mu\nu}(x) -\frac{i}{2}F_{\mu\nu}(x)\tilde F^{\mu\nu}(x).
\end{eqnarray}
We see that the component field Lagrangian, as expected and as promised in Sec.~\ref{sec:fieldstrength}, contains a field strength term for the vector component field. At this point we may worry about the appearance of the dual field strength tensor which we do not have in the Standard Model. However, we can show that the term in question can be written as a total derivative,
\begin{eqnarray}
F_{\mu\nu}\tilde F^{\mu\nu} &=& \half (\partial_\mu A_\nu-\partial_\nu A_\mu)\epsilon^{\mu\nu\rho\sigma}(\partial_\rho A_\sigma-\partial_\sigma A_\rho) \nonumber\\
&=& 2\epsilon^{\mu\nu\rho\sigma}(\partial_\mu A_\nu)(\partial_\rho A_\sigma)  \nonumber\\
&=& 2\epsilon^{\mu\nu\rho\sigma}(\partial_\rho\partial_\mu A_\nu) A_\sigma- 2\epsilon^{\mu\nu\rho\sigma}\partial_\rho(\partial_\mu A_\nu A_\sigma)
\end{eqnarray}
where the first term disappears due to the asymmetry of the Levi-Civita symbol. That means that the term with the dual field strength tensor disappears if the fields fall off rapidly enough towards infinity.  This will always be true for abelian field theories, but is not necessarily so for non-abelian theories, where we can have so-called instanton effects.
\end{Answer}


%%%
\begin{Exercise}[]
\label{ex:U1gaugegroup}
Extend Exercise \ref{ex:simpleL} to include a single abelian gauge group under which the scalar superfield has charge $q$. Simplify your answer using the covariant derivative $D_\mu\equiv \partial_\mu-i\frac{q}{2}V_\mu$.  {\it Hint:} The spinor relationship
\begin{equation}
\theta \psi \bar{\theta}\bar{\psi}(\theta\sigma^\mu \bar{\theta}) = -\frac{1}{4}(\theta\theta)(\bar\theta\bar\theta)\bar\psi\bar\sigma^\mu\psi,
\label{eq:spinoridentity_U1ex}
\end{equation}
may also come in handy.
\end{Exercise}

\begin{Answer}
We should note that in this case $W=0$ as we can not construct gauge invariant terms in the superpotential with only one superfield that is charged under the gauge group. This also implies that $F(x)\equiv0$.

For an abelian gauge group the kinetic term in Wess--Zumino gauge is
\begin{equation}
\mathcal L_\text{kin} =\int d^4\theta \,\Phi^\dagger e^{qV_{WZ}}\Phi=\int d^4\theta \, \left[\Phi^\dagger \Phi+ q\Phi^\dagger \Phi V_{WZ}+\frac{1}{2} q^2\Phi^\dagger \Phi V_{WZ}^2\right].
\end{equation}
The first term in the integral we have calculated already in Exercise~\ref{ex:simpleL}.  Inserting Eqs.~(\ref{eq:leftscalarsuperfield}), (\ref{eq:rightscalarsuperfield}) and (\ref{eq:vectorsuperfieldWZ}) the second term is
\begin{eqnarray}
\int d^4\theta \, \Phi^\dagger \Phi V_{WZ} &=&
 \int d^4\theta \, \left\{ [iA^*(\theta\sigma^\nu \bar{\theta})\partial_\nu A - iA(\theta\sigma^\nu \bar{\theta})\partial_\nu A^*+  2\bar{\theta}\bar{\psi}\theta \psi] (\theta\sigma^\mu \bar{\theta})V_\mu \right.   \nonumber \\
&&\left. + \sqrt{2}A \bar{\theta}\bar\psi  \theta\theta\bar{\theta}\bar{\lambda} + \sqrt{2}A^*\theta \psi \bar{\theta}\bar{\theta}\theta\lambda  +\theta\theta\bar{\theta}\bar{\theta}|A|^2D \right\}  \nonumber \\
&=& \int d^4\theta \, \left\{ \frac{i}{2} A^* \theta\theta\bar{\theta}\bar{\theta}\partial^\mu A V_\mu- \frac{i}{2}A \theta\theta\bar{\theta}\bar{\theta}\partial^\mu A^*V_\mu -\frac{1}{2}(\theta\theta)(\bar\theta\bar\theta)\bar\psi\bar\sigma^\mu\psi V_\mu \right.  \nonumber \\
&& \left. + \frac{1}{\sqrt{2}}A\theta\theta\bar\theta\bar\theta\bar\psi\bar\lambda +  \frac{1}{\sqrt{2}} A^*\theta\theta\bar\theta\bar\theta\psi \lambda +\theta\theta\bar{\theta}\bar{\theta}|A|^2D \right\} \nonumber \\
&=&  \frac{i}{2} ( A^*\partial^\mu A- A\partial^\mu A^* )V_\mu -\frac{1}{2}\bar\psi\bar\sigma^\mu\psi V_\mu \nonumber \\
&& + \frac{1}{\sqrt{2}}A\bar\psi\bar\lambda +  \frac{1}{\sqrt{2}} A^*\psi \lambda +|A|^2D,
\end{eqnarray}
where we have used Eqs.~(\ref{eq:Weylspinor_etapsietapsi}), (\ref{eq:Weylspinor_etasigmamuetaetasigmanueta}), and (\ref{eq:spinoridentity_U1ex})
for the second equality.

The third term is found from Eqs.~(\ref{eq:leftscalarsuperfield}), (\ref{eq:rightscalarsuperfield}) and (\ref{eq:V2_WZ}), with only one possible surviving term
\begin{eqnarray}
\int d^4\theta \, \half\Phi^\dagger \Phi V^2_{WZ} &=&\int d^4\theta \,  \frac{1}{4} |A|^2 \theta\theta\bar{\theta}\bar{\theta}V^\mu V_\mu = \frac{1}{4} |A|^2V^\mu V_\mu.
\end{eqnarray}

Summing up these three terms the total contribution from the kinetic terms is then
\begin{eqnarray}
\mathcal L_\text{kin} &=& -A^*\square A+i\partial_\mu\bar\psi\bar\sigma^\mu\psi +\frac{i}{2} q( A^*\partial^\mu A- A\partial^\mu A^* )V_\mu -\frac{1}{2}q\bar\psi\bar\sigma^\mu\psi V_\mu \nonumber \\
&& + \frac{1}{\sqrt{2}}qA\bar\psi\bar\lambda + \frac{1}{\sqrt{2}} qA^*\psi \lambda +q|A|^2D + \frac{1}{4}q^2 |A|^2V^\mu V_\mu.
\end{eqnarray}
We can simplify this expression by defining a covariant derivative from the vector component field $D_\mu\equiv \partial_\mu-i\frac{q}{2}V_\mu$.\footnote{The unusual factor of 2 here compared to the usual abelian gauge covariant derivative is due to our earlier choices in how to write down the vector superfield in terms of component fields. If we want we can absorb this by redefining $V_\mu$.}
We immediately see that 
\begin{equation}
i\partial_\mu\bar\psi\bar\sigma^\mu\psi-\frac{1}{2}q\bar\psi\bar\sigma^\mu\psi V_\mu =iD_\mu^*\bar\psi\bar\sigma^\mu\psi,
\end{equation}
and
\begin{eqnarray}
|D_\mu A|^2 &\equiv& D^*_\mu A^*D^\mu A= (\partial_\mu A^*+i\frac{q}{2}V_\mu A^*)(\partial^\mu A-i\frac{q}{2}V^\mu A) \nonumber\\
&=& |\partial_\mu A|^2+\frac{q^2}{4}V_\mu V^\mu |A|^2 +i\frac{q}{2}(A^*\partial^\mu A-\partial^\mu A^* A )V_\mu,
\end{eqnarray}
so that we can write, using $-A^*\square A= |\partial_\mu A|^2$,
\begin{eqnarray}
\mathcal L_\text{kin} &=& \frac{1}{\sqrt{2}}qA\bar\psi\bar\lambda + \frac{1}{\sqrt{2}} qA^*\psi \lambda +q|A|^2D 
+iD_\mu^*\bar\psi\bar\sigma^\mu\psi+|D_\mu A|^2 \nonumber
\end{eqnarray}

Adding the  field strength terms from Exercise \ref{ex:abelian_field_strength} the complete Lagrangian is
\begin{eqnarray}
\mathcal L &=& iD_\mu^*\bar\psi(x)\bar\sigma^\mu\psi(x)+|D_\mu A(x)|^2 -i\lambda(x)\sigma^\mu \partial_\mu \bar\lambda(x) \nonumber\\
&& + \frac{1}{\sqrt{2}}qA(x)\bar\psi(x)\bar\lambda(x) + \frac{1}{\sqrt{2}} qA^*(x)\psi(x) \lambda(x) +q|A(x)|^2D(x) 
\nonumber\\
&&+2D^2(x)-\frac{1}{4}F_{\mu\nu}(x)F^{\mu\nu}(x) .
\end{eqnarray}
\end{Answer}


%%%
\begin{Exercise}[]
\label{ex:SQED}
Find the Lagrangian for supersymmetric QED (SQED) in terms of component fields, eliminating any non-dynamical auxiliary fields. Supersymmetric QED is the smallest  (in terms of field content) supersymmetric theory that has the particles and interactions in QED as a subset.
\end{Exercise}

\begin{Answer}
In QED we have three particles, the electron, positron and the photon. As a Dirac fermion the electron needs two scalar superfields to provide the two different left- and right-handed Weyl-spinor components. However, then the positron is already given by the hermitian conjugates of these spinors. For the photon we need an abelian vector superfield for the $U(1)_{\text{em}}$ gauge group. Thus the minimum field content consists of two scalar superfields $L$ and $\bar E$, and a vector superfield $V$. 

The supergauge transformations for the scalar superfields are $L\to L'=\exp{(-ie\Lambda)}L$  and $\bar E\to \bar E'=\exp{(ie\Lambda)}\bar E$, where $e$ is the elementary electrical charge, and the difference in sign signifies the opposite charges of the electron and positron. The general form of the superpotential must then be $W=mLE$ with a single (mass) parameter $m$. No tadpole terms in $L$ and $\bar E$ can survive since these are not gauge singlets, and only the $LE$ mass term survives since the two fields in the mass term must have opposite charges under the gauge group. Similarly, no Yukawa term can fulfil the gauge invariance criterium. 

Solving for the auxiliary $F$-fields we  get
\begin{eqnarray}
F_L^* &=& -W_L = -\frac{\partial W[A_L,A_E]}{\partial A_L}= -mA_E, \nonumber \\
F_e^* &=& -W_E = -\frac{\partial W[A_L,A_E]}{\partial A_E}= -mA_L. \nonumber
\end{eqnarray}
Generalising the discussion in Exercise~\ref{ex:simpleL}, the contribution to the Lagrangian from the superpotential in terms of component fields is
\begin{equation}
\mathcal L_\text{superpot} = m\psi_L\psi_E+m^*\bar\psi_L\bar\psi_E-|m|^2(|A_L|^2+|A_E|^2),
\end{equation}
where we must reasonably identify $m$ with the electron mass, see also that, as expected the scalar fields $A_E$ and $A_L$ obtain the same mass.

From the field content we know that the kinetic terms take the form $L^\dagger e^{eV}L$ and $\bar E^\dagger e^{-eV}\bar E$ where the sign ensures the gauge invariance under the supergauge transformations of the scalar and vector superfields together. Taking results from Exercise~\ref{ex:U1gaugegroup} the Lagrangian contribution from the kinetic terms is 
\begin{eqnarray}
\mathcal L_\text{kin} &=&
\frac{1}{\sqrt{2}}eA_L\bar\psi_L\bar\lambda + \frac{1}{\sqrt{2}} eA^*_L\psi_L \lambda +e|A_L|^2D +iD_\mu^*\bar\psi_L\bar\sigma^\mu\psi_L+|D_\mu A_L|^2 \nonumber\\
&&-\frac{1}{\sqrt{2}}eA_E\bar\psi_E\bar\lambda - \frac{1}{\sqrt{2}} eA^*_E\psi_E \lambda - e|A_E|^2D +iD_\mu^*\bar\psi_E\bar\sigma^\mu\psi_E+|D_\mu A_E|^2, \nonumber
\end{eqnarray}
where $\lambda$ is the Weyl fermion from the vector superfield $V,$ $D$ is the auxiliary field from $V$, and the covariant gauge derivative is $D_\mu=\partial_\mu-\frac{ie}{2}A_\mu$, where $A_\mu$ is the electromagnetic gauge field.

In Exercise~\ref{ex:abelian_field_strength} we saw that the contribution from the abelian field strength term is
\begin{eqnarray}
\mathcal L_\text{field strength}= -i\lambda\sigma^\mu \partial_\mu \bar\lambda+2D^2-\frac{1}{4}F_{\mu\nu}F^{\mu\nu},
\end{eqnarray}
where $F_{\mu\nu}=\partial_\mu A_\nu-\partial_\nu A_\mu$.

Solving for the auxiliary $D$-field we get
\begin{equation}
D=\frac{e}{4}(|A_E|^2-|A_L|^2).
\end{equation}

This gives a total Lagrangian of
\begin{eqnarray}
\mathcal L &=&  iD_\mu^*\bar\psi_L\bar\sigma^\mu\psi_L+|D_\mu A_L|^2 +iD_\mu^*\bar\psi_E\bar\sigma^\mu\psi_E+|D_\mu A_E|^2 -i\lambda\sigma^\mu \partial_\mu \bar\lambda-\frac{1}{4}F_{\mu\nu}F^{\mu\nu} \nonumber\\
&& +m\psi_L\psi_E+m^*\bar\psi_L\bar\psi_E-|m|^2(|A_L|^2+|A_E|^2) \nonumber \\
&& + \frac{1}{\sqrt{2}}eA_L\bar\psi_L\bar\lambda + \frac{1}{\sqrt{2}} eA^*_L\psi_L \lambda
-\frac{1}{\sqrt{2}}eA_E\bar\psi_E\bar\lambda - \frac{1}{\sqrt{2}} eA^*_E\psi_E \lambda - \frac{e^2}{8}(|A_E|^2-|A_L|^2 )^2.\nonumber
\end{eqnarray}
\end{Answer}


\end{document}
