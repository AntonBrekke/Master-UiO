\documentclass[12pt,a4paper]{article}
\usepackage{amsmath, amsfonts, amssymb, fancyhdr, floatflt, epsfig, graphicx,psfrag}
\usepackage[utf8]{inputenc}
\usepackage{newtxtext,newtxmath}
\usepackage{multicol}
\usepackage{xcolor}
\usepackage{url}

%A4 = 210x297 => 170x257 with 2cm margins
%=>these are generous ways of implementing this
\textwidth=16cm
\textheight=24.5cm

%now adjust TeX margins to generously fit this 
\topmargin=-0.3in
\oddsidemargin=0.0cm
\evensidemargin=0.0cm

\usepackage{hyperref}
%\hypersetup{
%    colorlinks=true,
%    linkcolor=blue,
%    filecolor=magenta,      
%    urlcolor=cyan,
%}

\newcommand{\ds}{{\sf DarkSUSY}}

\begin{document}
\pagestyle{plain}


 \vspace*{-1.5cm}

\thispagestyle{empty}

{\begin{center}
MSc project \\[1ex]
 {\large \bf  Thermal production of feebly interacting dark matter}\\[0.5ex]
 {\scriptsize \bf Kristian Gjestad Vangsnes}
\end{center}
}
 \vspace*{0.5cm}

\noindent{\bf Introduction}\\[1ex]
There is compelling evidence for the existence of dark matter (DM), where the only effects we observe are gravitational. 
We see these effects in the  rotational curves of spiral galaxies, gravitational lensing and importantly in the cosmic 
microwave background (CMB), where the inferred cosmological DM abundance is about five times the abundance of ordinary 
matter~\cite{Planc_colab}.  The most popular scenario for the creation of DM is the so-called freeze-out mechanism,
where DM was in thermal equilibrium with the visible sector heat bath at early times~\cite{Gondolo:1990dk};
once its annihilation rate fell behind the expansion rate of the universe, DM would decouple and its comoving number 
density freeze to a constant value. 
A typical example for this are so-called weakly interacting massive particles (WIMPs) 
with mass around the electroweak scale. However there is no theoretical reason why the assumption that the 
DM was in thermal equilibrium with the visible heat bath must hold. Instead the present DM abundance may have 
been generated out of equilibrium, in a process known as freeze-in mechanism. In this scenario, DM couples very weakly, 
which means that it never reached thermal equilibrium. Instead, the DM particles were continuously produced by decay or 
annihilation processes from the visible sector -- or a dark sector heat bath -- until  the  production  stopped once  the  heat bath 
temperature dropped  below the relevant mass scale connecting the DM particle to the visible sector.  
Due to the small 
coupling strength, DM particles produced via the freeze-in mechanism have been called Feebly Interacting 
Massive Particles (FIMPs)~\cite{FIMP_DM_Summary}. 


 \vspace*{0.6cm}
\noindent
{\bf Scope}\\[1ex]
The FORTRAN package \ds~\cite{DarkSUSYsite} was originally developed to numerically calculate properties of 
supersymmetric DM, which historically was the most popular DM candidate. The package has been upgraded to include 
other theories containing DM candidates as well~\cite{DarkSUSY}, which also will be the focus of my MSc project. 
Its main scope will be to further develop \ds\ by adding the possibility to calculate the freeze-in abundance of FIMP 
DM candidates.
As of today, the only publicly available code with this ability is {\sf micrOMEGAs5.0}~\cite{micrOMEGAs5}. An important 
task will thus be to independently verify the results presented there, and check the validity of the approximations
entering in their derivation. In a second step, the newly written \ds~routines will be used to study freeze-in production
for a model where DM does not couple to standard model particles, but only to particles contained in a separate, `dark'
sector that is thermalized but with a temperature different from that of the photons.


%\newpage 
 \vspace*{0.6cm}
\noindent
{\bf Project tasks}\\
{\scriptsize (dates mark deadlines for the completion of the respective task)}\\[-3ex]

\begin{itemize}
\item Thorough familiarization with the theoretical framework of how to calculate the DM relic density, both for freeze-out and 
freeze-in, as part of a special reading course in the curriculum. {\scriptsize \bf [\emph{June 2020}]}

\item Familiarization with programming in FORTRAN, and the structure of \ds. Demonstrate basic usage of {\sf git} 
(the version control system \ds\ is based on), the makefiles used in \ds, and \ds\ example programs that involve 
{\it i)} different particle modules and {\it ii)} `replaceable functions'. 
{\scriptsize \bf [\emph{July 2020}]}

\item Write a first version of subroutines to calculate the relic density, and a main program to test it, focussing on
the simpler case of assuming Maxwell-Boltzmann statistics for all particles involved. Specifically, test the 
results obtained in Ref.~\cite{micrOMEGAs5} for a $Z'$ portal. \\{\scriptsize \bf [\emph{September 2020}]}

\item Update the routines to use the full Bose-Einstein statistics and Fermi-Dirac statistics, 
and compare the results with those reported for the Scalar portal in Ref.~\cite{micrOMEGAs5}. 
{\scriptsize \bf [\emph{November 2020}]}


\item Tidy up and optimize existing code. Prepare a short written report of results obtained so far, including the 
theoretical tools necessary to obtain them. {\scriptsize \bf [\emph{December 2020}]}

\item Implement dark sector model with changing number of degrees of freedom (and therefore dark sector temperature).
Compute freeze-in production and discuss differences to models studied so far.
 {\scriptsize \bf [\emph{February 2021}]}


\item Fully focus on thesis writing  {\scriptsize \bf [ \underline{from} \emph{Februrary -- March 2021}]}
\end{itemize}




% \vspace*{0.3cm}
%\noindent
%{\bf Methods}
% \vspace*{0.1cm}

%The first part  of the project largely consists in literature research and the compilation of a 
%short written report. The following steps then require mostly analytical calculations, 
%but likely numerical evaluations of the final results (both to be done with the  {\sf mathematica} package).

 \vspace*{0.5cm}
{
%\footnotesize
%\noindent
%{\bf References}

\renewcommand\refname{\normalsize \bf References}
\bibliographystyle{unsrt}

\bibliography{refs.bib}


\end{document}